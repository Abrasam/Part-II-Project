\documentclass[12pt,notitlepage,a4paper]{report}
%TC:group tabular 1 1
%TC:group table 1 1
\usepackage{verbatim}
\usepackage[hidelinks]{hyperref}
\usepackage{tabularx}
\usepackage{graphicx}
\usepackage{algorithm}
\usepackage{algpseudocode}
\usepackage{cleveref}
\usepackage{dirtree}
\usepackage{xcolor}
\usepackage{wrapfig}
\usepackage{subcaption}
\usepackage[margin=2.5cm]{geometry}
\usepackage{standalone}

\crefname{section}{§}{§§}
\Crefname{section}{§}{§§}

\raggedbottom                           % try to avoid widows and orphans
\sloppy
\clubpenalty1000%
\widowpenalty1000%

\renewcommand{\baselinestretch}{1.1}    % adjust line spacing to make
% more readable

\renewcommand\DTstyle{\ttfamily\textcolor{blue}}
\renewcommand\DTstylecomment{\ttfamily\textcolor{gray}}

\setcounter{secnumdepth}{4}
\setcounter{tocdepth}{4}
\setlength{\parskip}{0.5em}%

\newcommand{\python}{\emph{Python}}
\newcommand{\unity}{\emph{Unity}}
\newcommand{\kademlia}{\emph{Kademlia}}
\newcommand{\K}{$k$}
\newcommand{\voxpop}{\emph{VoxelPopuli}}

\begin{document}
	%TC:ignore
	\bibliographystyle{plain}
	
	\pagestyle{empty}
	
	\hfill{\LARGE \bf Samuel J. Sully}
	
	\vspace*{60mm}
	\begin{center}
		\Huge
		{\bf \voxpop:\\ A Decentralised Peer-to-Peer Voxel-Based World} \\
		\vspace*{5mm}
		Computer Science Tripos \\
		\vspace*{5mm}
		Robinson College \\
		\vspace*{5mm}
		2019-20
	\end{center}
	
	\clearpage
	
	\setcounter{page}{1}
	\pagenumbering{roman}
	\pagestyle{plain}
	
	\newpage
	\section*{Declaration}
	
	I, Samuel John Sully of Robinson College, being a candidate for Part II of the Computer
	Science Tripos, hereby declare that this dissertation and the work described in it are
	my own work, unaided except as may be specified below, and that the dissertation does
	not contain material that has already been used to any substantial extent for a comparable
	purpose.
	
	\bigskip
	
	\noindent I, Samuel John Sully of Robinson College, am content for my dissertation to be made available to the students and staff of the University.
	
	\clearpage
	
	\chapter*{Proforma}
	{
		\begin{tabular}{ll}
			Candidate Number:   & \bf 2348D\\
			Project Title:      & \bf Voxel Populi: A Decentralised Peer-to-Peer Voxel-Based\\ 
			& \bf World\\
			Examination:        & \bf Computer Science Tripos -- Part II, July 2020\\
			Word Count:         & \bf 10271\footnotemark[1]\\
			Line Count:         & \bf 1860\footnotemark[2]\\
			Project Originator: & \bf Self\\
			Supervisor:         & \bf Prof. Jon Crowcroft\\
		\end{tabular}
	}
	\footnotetext[1]{Computed by {\tt texcount -1 -utf8 -sum -inc dissertation.tex}. Tables included, all content outside of chapters $1-5$ excluded.}
	\stepcounter{footnote}
	\footnotetext[2]{Computed by {\tt cloc --exclude-dir=\_\_pycache\_\_ --exclude-lang=XML VoxelPopuli/Code}.}
	\stepcounter{footnote}
	
	\section*{Original Aims of the Project}
	My project aimed to create a peer-to-peer 3D world using a distributed hash table (DHT), namely \kademlia{}~\cite{kademlia}. I aimed to explore this approach for Massively Multiplayer Online games (MMOs) to see if it is viable. The decentralised approach offers various advantages, such as better load balancing and game longevity.
	
	\section*{Work Completed}
	I implemented a distributed 3D virtual world suitable for MMOs and large scale simulations. The three parts of my project are all functioning correctly. I implemented \kademlia{} with some modifications; I implemented the game server to run above the DHT and process the computation for a set of chunks of the world and I implemented the graphical client in \unity{} which connects to the world and allows a user to move around and interact with it. My test client demonstrated the viability of my project at scale.
	
	\section*{Special Difficulties}
	None.
	
	\tableofcontents
	
	\listoffigures
	
	\listoftables
	
	\listofalgorithms
	
	\newpage

	\clearpage
	
	\setcounter{page}{1}
	\pagenumbering{arabic}
	\pagestyle{headings}
	%TC:endignore
	\chapter{Introduction}
	
	\section{Project Summary}
	My project, \voxpop{}, explores a peer-to-peer architecture for Massively Multiplayer Online Games (MMOs) or large scale simulations. This is in contrast to the more commonly used centralised approach. \voxpop{} is built upon a distributed hash table which is used to locate in the peer-to-peer network the server responsible for handling any particular part of the world.
	
	\voxpop{} consists of three parts: the distributed hash table which is a modified version of the \kademlia{}~\cite{kademlia} specification; the game server which runs the computation for certain segments of the world and the \unity{} client used to interact with the world. All these have been completed in adherence to the success criteria in my project proposal, as well as the evaluation client used in the evaluation stage. A large scale test of \voxpop{} using my dedicated server demonstrated the project's viability and provided useful data to inform future iterations.
	
	\section{Motivation}
	The MMO genre is very popular\footnote{\emph{World of Warcraft} -- a popular MMO -- had 7.7 million subscribers in 2019.} in modern gaming, as an increasing proportion of the populace have access to high-speed broadband the prevalence of these games continues to increase. Most of these games employ a centralised client-server mode where the creators of the MMO have a relatively small number of expensive and powerful machines which they use to handle all players.
	
	This centralised approach often requires some form of `sharding'~\cite{shard}, whereby players are allocated into separate, independent instances (`shards') of the same world. This means that players can only interact with others connected to the same shard. The centralised approach also means that the game creators have total authoritative control over the game.
	
	An alternative approach is a decentralised, peer-to-peer approach which I explore in this project. In this approach the world is separated into segments (or `chunks') and each peer in the network is responsible for handling the load for a number of chunks. This approach implicitly performs load balancing and is highly failure tolerant, as a node failure can be dealt with by simply having another take over.
	
	This has a number of advantages over the centralised, sharded approach. One significant advantage is that the world is able to be explicitly mutable (such as the voxel-based world I have implemented), with the sharded approach if a player makes a change in one shard then we may need some way of propagating these changes to the other shards while maintaining consistency. However, in my approach there is only one server which is authoritative for the state of any part of the world so there is no need for complex consensus mechanisms.
	
	A further advantage is that the system has improved longevity. When large-scale MMOs cease to be profitable or useful for the developers, who operate the centralised servers, they often shut them down, as recently happened with the popular MMO \emph{Club Penguin}~\cite{clubpenguin} in 2017. With my approach, if we allow individuals to create their own servers to join the peer-to-peer network then, provided there exists a community dedicated to keeping the MMO running, it can continue to exist at no cost to the developers. It would even be possible to have multiple, separate networks running or even networks running modified versions of the game.
	
	\section{Related Work}
	There are very few large-scale, peer-to-peer MMOs, likely due to the security issues I explore in the evaluation chapter and due to the fact that it limits the ability for the developers to monetise the MMO post-release. However, it is possible that techniques similar to mine may be used behind the scenes on a number of large-scale MMOs.
	
	One similar piece of work is \emph{SpatialOS}~\cite{SpatialOS}, this is a platform for managing online games or simulations in the cloud. It works in a similar way to \voxpop{}, by splitting up the world into segments which are administrated by separate servers. \emph{SpatialOS} is produced by the startup Improbable and is still fairly new, however, it is being used in the development of a number of games.
	
	It is worth noting also that while my implementation of \kademlia{} is custom, I used a \kademlia{} library~\cite{pykademlia} for \python{}, called \emph{PyKademlia}, as a reference for a fully functioning Kademlia implementation. The approach used in \emph{PyKademlia} follows a slightly different, later version of the \kademlia{} specification, I follow the original specification outlined in the cited paper.
	
	\clearpage
	\chapter{Preparation}
	
	\section{Starting Point}
	Prior to this project I had limited experience in implementing distributed systems, my knowledge on such systems mainly comes from the Part IB courses Concurrent and Distributed Systems and Computer Networking. Computer Networking introduced the concept of distributed hash tables (DHTs) which are used extensively in \voxpop{}. Concurrent and Distributed Systems introduces most of the overarching principles of distributed systems, such as RPCs, which are essential in \voxpop{}. Furthermore, this project relies on knowledge from a number of other courses, such as Part II Principles of Communication and Part IA Introduction to Graphics. I have some limited experience with 3D graphics from my own hobby programming as well.
	
	\section{Requirement Analysis}
	\label{sec:req}
	\voxpop{} aims to implement a suite of software for the operation, interaction with and testing of a 3D world which is distributed over a number of peers in a peer-to-peer network. The success criteria set out in my proposal are as follows:
	
	\begin{enumerate}
		\item My DHT must adhere to the \kademlia{} specification~\cite{kademlia}. It is possible I will need to make some changes to fit the specification better to my needs and this is acceptable.
		\item The peer-to-peer node program must join the network, bootstrapping via some known node, and then will be able to participate in hosting the game world as it becomes part of the DHT.
		\item It must be possible to interact with the world using a simple 3D graphical client, which is able to place and remove voxels from the world. These changes must persist.
		\item The system must handle player moving between separate chunks (and thus, separate peers) seamlessly, with no loading screen.
		\item There must be a simple test agent which connects to and interacts with the world in some notional way to emulate the behaviour of a human user. This is for the purposes of quantitative evaluation.
	\end{enumerate}

	In addition to these, the project needs to fulfil a number of other requirements:
	
	\begin{itemize}
		\item \textbf{Robustness:} the system must be very robust, handling node failures with minimal disruption to the overall system, minimising disruption to users connected to the system at a given time.
		\item \textbf{Deployment:} the implementation must run as a cloud application, being easily deployable to a large number of machines. In my testing I will use my dedicated server running \emph{Ubuntu 18.04}.
		\item \textbf{Decentralisation:} the implementation must be designed to be entirely decentralised, nodes in the P2P network must be entirely equal, there must be no authoritative entity in the system.
		\item \textbf{Mutability:} the game world must emulate that of voxel-based games such as \emph{Minecraft}. As such, users must be able to edit the world and have these changes persist, users' positions must also be stored so that when they log out and back in at another time (or to a different server), they return to where they left off.
	\end{itemize}

	\section{Kademlia}
	\voxpop{} is built using a DHT at its core, a DHT is a decentralised storage system based on the commonly used hash table data structure. DHTs store $\langle$ key,value$\rangle$~pairs, these are distributed among the nodes in the network, with there existing some method to partition the set of keys between the nodes, preferably in such a way that node joins or leaves require minimal changes to this partition (i.e. a node leaving does not cause the entire key-node mapping to change). The DHT maintains an \emph{overlay network} where each node maintains a set of links to other nodes in the DHT according to the topology of the network, this set of links is used in routing queries around the DHT.
	
	\subsection{XOR Metric}
	The \kademlia{} specification sets out that identifiers be $160$bit integers. Nodes IDs and keys for the DHT occupy this ID space. The notion of distance between identifiers, $d(x,y)$, is given by the bitwise XOR of the two interpreted as an integer\footnote{This is effectively a derivation for the common prefix length of the two IDs, see the second answer given here: \url{https://stackoverflow.com/questions/25751928/kademlia-xor-metric-properties-purposes/25756389} (accessed 2020-05-01.)} (i.e. $d(x,y) = x \oplus y$). This is a valid metric as it obeys the following properties:
	\begin{enumerate}
		\item $d(x,x) = 0$, that is, the distance from any identifier to itself is $0$.
		\item $d(x,y) > 0$ if $x \neq y$, that is, the  distance between any two distinct identifiers is larger than $0$.
		\item $d(x,y) = d(y,x)$, that is, distances are symmetric.
		\item Distances obey  the triangle inequality, i.e. $d(x,z) \leq d(x,y) + d(y,z)$.
	\end{enumerate}
	The set of keys which a node `owns' is given by all those which are closest to its ID using the above notion of distance\footnote{This is not strictly true, actually the \K~closest nodes all store values for that key, where \K~is a parameter of the network.}.
	
	\subsection{Node State}
	Each node maintains some amount of information about other nodes in the network in order to route messages. Each node maintains a \K-bucket for each $i$ in $0 \leq i < 160$ , a \K-bucket is simply a sorted list (of length \K) of $\langle$ IP address, UDP port, node ID$\rangle$~triples of nodes from $2^i$ to $2^{i+1}-1$ (inclusive) distance away from this node. The lists are sorted by time last seen, such that the most recently seen node is at the tail of the list. This is useful later when evicting stale nodes from the \K-bucket. Note that \K~is a parameter of the network, the replication parameter.
	
	In order to populate these \K-buckets, whenever a node receivers a message from another, it looks for the appropriate \K-bucket and, if the sender is already in the \K-bucket then it is moved to the tail of the list, otherwise it is appended to the tail of the list. If the \K-bucket is full then we send a \texttt{PING} RPC to the least recently seen node, if it fails to reply then we evict it and put the new node in instead, else we discard the new node\footnote{In my implementation, the new node is added to a queue to join the \K-bucket.}.
	
	\subsection{RPCs}
	The Kademlia protocol has four RPCs: \texttt{PING}, \texttt{FIND\_NODE}, \texttt{FIND\_VALUE} and \texttt{STORE}. All other operations are built up from these four RPCs. Table~\ref{tab:rpcs} details the function of each RPC. My implementation deviates from this specification as detailed in \cref{sec:customrpc}.
	
	\begin{table}[!ht]
		\begin{tabularx}{\textwidth}{| l | X |}
			\hline
			\texttt{PING} & Used to check whether a node is online, upon receiving a \texttt{PING} RPC a node will reply with its ID.\\
			\hline
			\texttt{FIND\_NODE} & Takes a $160$bit integer as argument (and identifier). When a node receives a \texttt{FIND\_NODE} RPC it returns $\langle$ IP address, UDP port, node ID$\rangle$~ triples from the \K nearest nodes to the argument identifier that it knows of.\\
			\hline
			\texttt{FIND\_VALUE} & Behaves in the same was as \texttt{FIND\_NODE} but will return a value if it possesses one for the supplied ID.\\
			\hline
			\texttt{STORE} & Takes a $\langle$ key, value$\rangle$~ pair which the receiving node stores.\\
			\hline
		\end{tabularx}
		\caption{The four \kademlia~RPCs.}
		\label{tab:rpcs}
	\end{table}
	
	\subsection{Node Lookup}
	The lookup procedure is used to locate the \K~closest nodes to a supplied identifier. The lookup procedure has one parameter, the concurrency factor $\alpha$. It proceeds as follows:
	\begin{enumerate}
		\item Find $\alpha$ closest nodes from own \K-buckets.
		\item Send \texttt{FIND\_NODE} RPCs to these $\alpha$ nodes searching for supplied identifier.
		\item Then we recursively send \texttt{FIND\_NODE} requests nodes it learned of from the results of previous steps.
		\item When an iteration of RPCs gives us no new nodes better than the current closest, we send RPCs to all of the \K~closest nodes we have not yet queried.
		\item The procedure terminates when we have received a response from all of the \K~nearest nodes.
	\end{enumerate}
	The \K~nearest nodes are returned from this procedure.
	
	\subsection{Value Lookup}
	The procedure for retrieving a value from the DHT is similar to the node lookup procedure above, replacing the \texttt{FIND\_NODE} RPCs in the above description with \texttt{FIND\_VALUE} RPCs. Instead of returning the \K~nearest nodes it will return the value found, or some notional \texttt{NULL} value if none exists.
	
	\subsection{Value Storage}
	The store value procedure consists of performing a lookup node procedure as above with the identifier being the key of the $\langle$ key, value$\rangle$~ pair to be stored. Then \texttt{STORE} RPCs with the $\langle$ key, value$\rangle$~ pair are sent to the \K~nodes returned from the lookup.
	
	\subsection{Bootstrap}
	Bootstrapping is the process by which a node joins the network. Because \kademlia{} routing information is implicitly learned through network activity we do not need an explicit \texttt{JOIN} method, we can simply use existing RPCs to join a network. All that we need is the IP, port and ID of any existing node in the network, this is the bootstrap node.
	
	The joining node, $n$, inserts the bootstrap node, $m$, into the appropriate \K-bucket and then performs a node lookup for its own ID. Finally it refreshes all its buckets which are further away than its closest neighbour. Refreshing a \K-bucket simple means picking a random ID from that bucket's range and performing a node lookup for that ID. This operation is performed automatically by each node periodically on all buckets which have not been touched in a certain amount of time\footnote{Usually 1 hour.}. By performing a lookup of itself and by refreshing those \K-buckets we have ensured that this node has been inserted into the routing tables of a number of other nodes.
	
	\section{Game Server}
	The second major part of \voxpop{} is the game server, for this I use an architecture similar to that used by \emph{Minecraft} and by \emph{Valve}'s \emph{Source} engine~\cite{source}. An instance of a game server is the authoritative dedicated host that runs the computations for a given set of chunks of the game world, a client connects to a number of servers in order to receive the current world state and display it to the user graphically. This section of the system is purely client server, clients do not communicate among one another, instead doing so via the server(s).
	
	The server uses an approach used in both \emph{Minecraft} and \emph{Source} where the game world is simulated in discrete time steps known as `ticks'. During a tick we process any incoming packets and update the state of the world, then we send any packets to clients in order to update the world state. In these examples world state is transferred to clients using \emph{delta compression}, where, after the initial sending of the game state, we only send changes that happened since the last tick, this reduces network load.
	
	A number of further approaches could be employed by my implementation, such as compensating for latency and interpolating between ticks. However, these are beyond the scope of my investigation and are thus not a requirement.

	\section{Client}
	The third major part of \voxpop{} is the client, which is used to connect to and interact with, the world. This section of the project requires some 3D graphics, thus it draws on material from the two graphics courses in Part IA and Part IB. The client also implements the algorithm for locating and loading the relevant chunks into the world so that the chunks surrounding the player's current position are always loaded.
	
	For this section of the project I decided to use \unity, rather than \emph{LWJGL}, because the graphical element was simpler and as graphics is not the focus of my investigation this felt appropriate.
	
	\subsection{\unity}
	\unity{} is a 3D game engine which is widely used. It is fairly easy to learn yet quite powerful and expressive making it a popular tool in the modern games industry. A simple overview is all that is needed for this project. \unity{} provides us with a 3D world populated with game objects. These game objects can have \emph{components} attached to them, such as a mesh renderer (to render a 3D mesh) or light source for example. However, most importantly you are able to assign scripts as components which you can write yourself, these scripts have a number of built-in methods which \unity{} calls at particular times, most importantly we have the {\tt Start()} method, which is called upon creating the component, the {\tt Update()} method which is called each frame and the {\tt FixedUpdate()} method which is called each time the physics engine updates. These scripts have access to a powerful API allowing us to influence the game world.
	
	\section{World \& Terrain}
	The game world is analagous to that of \emph{Minecraft}, in that it consists of voxels (i.e. blocks) arranged in a 3D grid. An example of \emph{Minecraft}'s terrain can be seen in figure~\ref{fig:minecraft}. The \emph{Minecraft} world is broken into `chunks' each $16\times16\times256$ blocks, then each chunk is simply a 3D array of block data.
	\begin{figure}[!ht]
		\includegraphics[width=\textwidth]{minecraft.png}
		\caption{A screenshot of terrain from the game \emph{Minecraft}.}
		\label{fig:minecraft}
	\end{figure}

	The terrain in \emph{Minecraft} is generated procedurally, allowing for infinite worlds to be created on the fly. A common approach in procedurally generated video games is to use some form of coherent noise\footnote{Coherent noise simply means smooth pseudorandom noise which obeys the following properties: \begin{enumerate} \item The same input always gives the same output. \item A small change in the input produces a large change in the output. \item A large change in the input produces a random change in the output.\end{enumerate}} to generate a height map\footnote{Simply a 2D function or array where the value at any given point is the height of the terrain at any given point.}. I plan to use Perlin noise~\cite{perlin} (or its successor, Simplex noise~\cite{simplex}) to generate a heightmap for my world. Then I use simple rules to assign blocks at different heights different values (i.e. grass on top, followed by dirt, followed by stone) in order to produce a \emph{Minecraft}-like world. The structure of the world into chunks allows for easy segmentation across servers as each chunk can reside on a different server, additionally, by having data represented within a chunk as a 3D array this makes editing the world simple.
	
	\section{Professional Practice}
	\subsection{Ethical Implications} %pron - persistend customisable environment
	One legal and ethical concern is that \voxpop{} would give users access to a canvas within which they could, theoretically, encode any data. This could give rise to potential misuse by others. An interesting example is the use of \emph{Minecraft} to smuggle censored journalism into states with strict censorship laws~\cite{verge}.
	
	\subsection{Methodology}
	The project was broken up into discrete features, with a timeline planning to complete each in approximately two to three weeks. Thus I followed the \emph{Agile} software development workflow. Each \emph{sprint} had a deliverable which could be tested independently and demonstrated. My sprint timetable outlined in the proposal was adapted as the projected moved forward and some parts of the project took more, or less, time than anticipated.
	
	\subsection{Tooling}
	I used the \emph{PyCharm} IDE for the development of the \kademlia{} implementation and my game server as these were both written in Python using version $3.8$ due to improvements made to the \emph{asyncio} library in Python $3.8$. For the client I used \unity{} with \emph{Microsoft Visual Studio 2017} for editing the \emph{C\#} scripts. \emph{Git} was used for version control, with code pushed to \emph{GitHub} regularly and further backed-up daily to both the SRCF\footnote{Student-Run Computing Facility.} and the MCS using a \emph{cron} job.
	
	\clearpage
	\chapter{Implementation}
	\voxpop{} consists of three parts: the bespoke \kademlia{} implementation, the game server and the client. The system works by having the client query the \kademlia{} implementation to locate the appropriate servers for a particular part of the world, then connecting to that server and `joining' the world via that server. This is visualised in figure~\ref{fig:diagram1}.
	
	\begin{figure}[!ht]
		\includegraphics[width=\textwidth]{diag1.png}
		\caption{Diagram giving an overview of \voxpop{} architecture.}
		\label{fig:diagram1}
	\end{figure}
	The client needs to connect to a single node from the \voxpop{} network, which it will use as a `stepping stone' to access the whole network via. It uses this stepping stone node to query the DHT to find the $\langle$ IP, port$\rangle$~of the game servers responsible for the chunks it needs. It then initiates connections with each of these game servers in parallel, registers the player in that chunk and downloads the world data.
	
	The \voxpop{} server consists of two distinct parts, the \kademlia{} node and the game server. Thus for each node in the network there are two \emph{virtual} nodes. The \kademlia{} nodes are not visible to a client. In order to query the DHT, clients connect to any game server and initiate a special type of session specifically for DHT access. The game server then performs queries to the DHT on behalf of the client. Further details of this special session are in \cref{sec:interface}. Note that for simplicity's sake the \kademlia{} node and game server bind to the same IP address with the game server port being the \kademlia{} port incremented by one.
	
	\section{Kademlia}
	This section introduces my bespoke implementation of the \kademlia{} specification. A custom implementation of \kademlia{} was necessary for two main reasons:
	\begin{itemize}
		\item The project necessitated not only \texttt{PUT} and \texttt{GET} procedures but also a \texttt{GENERATE} procedure for locating the appropriate server for a particular chunk and then instructing it to generate said chunk.
		\item Two distinct types of data needed to be stored in the network. The network needed to store chunk location information (IP and port of appropriate server) and player state information in distinct areas so as not to conflate them.
	\end{itemize}
	As such, a custom implementation was devised with additional RPCs and a bespoke high-level interface with the required \texttt{PUT}, \texttt{GET} and \texttt{GENERATE} procedures.
	
	\subsection{RPC Framework}
	\label{sec:rpcframework}
	In order for \kademlia{} nodes to communicate with each other we need a method of issuing RPCs to remote machines and retrieving the results. In order to do this I used \emph{asyncio}'s \texttt{DatagramProtocol} class which is a base class for implementing protocols over UDP. This class provides overrideable methods such as \texttt{datagram\_received()} which is called when the underlying socket receives a UDP datagram. I wrote a general-purpose RPC system because at the time I was unsure how many RPCs I would need and wanted to be able to add and remove them on the fly. This proved useful when revising my \kademlia{} implementation to include the separate \texttt{STORE} and \texttt{FIND\_VALUE} RPCs.
	
	In order to implement the RPC framework I designed a \emph{JSON} format for RPC calls and responses (see table~\ref{tab:rpcjson} for details). I created two function decorators in Python: \texttt{@rpc} and \texttt{@stub}. \texttt{@rpc} adds no special behaviour and simply marks that this method may be called remotely; \texttt{@stub} replaces the supplied method with a method which takes the same arguments and computes the JSON RPC string to be sent to the other machine, then sends this message and awaits a reply before returning the result (or \texttt{None} on a timeout), this process is outlined in algorithm~\ref{alg:rpcstub}.
	\begin{table}[!ht]
		\begin{tabularx}{\textwidth}{| l | X |}
			\hline 
			\texttt{id} & 32-bit random number to uniquely identify this RPC call.\\
			\hline
			\texttt{node} & ID of the sending node.\\
			\hline
			\texttt{call} & Boolean representing whether this is an RPC call or a response.\\
			\hline
			\texttt{rpc} & The name of the remote procedure to be called.\\
			\hline
			\texttt{args} & The list of arguments (in order) to be supplied to the remote procedure. Not present on responses.\\
			\hline
			\texttt{ret} & The return value of a remote call. Not present on calls.\\
			\hline
		\end{tabularx}
	\caption{JSON RPC specification.}
	\label{tab:rpcjson}
	\end{table}

	Upon receiving a datagram the RPC framework decodes it as a JSON string and checks whether it is a function call or a response. In the case that it is a call, the relevant function is checked to determine if it has the \texttt{@rpc} decorator (i.e. it has been marked for remote calling) and then executes it, packaging the result up as a JSON message and returning it to the sender. If it is a response, it checks if there is a pending RPC with that ID, if so it will supply the result to that RPC call, otherwise the message is discarded. This process is detailed in algorithm~\ref{alg:datagram}.
	\begin{algorithm}[!ht]
		\begin{algorithmic}
			\Function{rpc\_stub}{func, to\_node, args}
			\State {id $\gets$ randombits(32)}
			\State {json $\gets$ \{``id'': id, ``node'': this\_node\_id, ``call'': true, ``rpc'': func, ``args'': args\}}
			\State {store pending rpc in table}
			\State {send\_udp(json, to)}
			\State {schedule timeout}
			\State {await response OR timeout}
			\If{timed out}
			\State \Return None
			\Else
			\State {response $\gets$ get\_response(id)}
			\State \Return response
			\EndIf
			\EndFunction
		\end{algorithmic}
		\caption{RPC framework \texttt{@stub} decorator algorithm.}
		\label{alg:rpcstub}
	\end{algorithm}
	\begin{algorithm}[!ht]
		\begin{algorithmic}
			\Function{datagram}{data, from}
				\State {msg $\gets$ json\_decode(data)}
				\If{msg.get(``call'')}
					\State func $\gets$ get\_function(msg.get(``RPC''))
					\If {func exists AND func has decorator \texttt{@rpc}}
						\State {res $\gets$ func(msg.get(``args''))}
						\State {json $\gets$ \{``id'': msg.get(``id''), ``node'': this\_node\_id, ``call'': false, ``rpc'': func.name, ``ret'': res\}}
						\State {send\_udp(json, from)}
					\EndIf
				\Else
					\If{rpc with id msg.get(``id'') is pending}
						\State {send msg.get(``ret'') to pending RPC calls}
					\EndIf
				\EndIf
			\EndFunction
		\end{algorithmic}
		\caption{Datagram handling in my JSON RPC framework.}
		\label{alg:datagram}
	\end{algorithm}
	
	This framework allows for the \kademlia{} specification to be implemented as described in the following sections. It allows methods to be tagged as remotely callable and allows for the creation of \emph{stub} methods which allow calling of RPCs on remote machines. This is necessary for the implementation of the \kademlia{} RPCs.
	
	\subsection{Custom RPC Specification}
	\label{sec:customrpc}
	In table~\ref{tab:rpcs} I outlined the RPCs in the default \kademlia{} specification. In my implementation I separated \texttt{STORE\_PLAYER} and \texttt{STORE\_CHUNK} RPCs as well as equivalent variants of the \texttt{FIND\_VALUE} RPCs. These replace the default \texttt{STORE} and \texttt{FIND\_VALUE} RPCs meaning we have a final specification consisting of six RPCs: \texttt{PING}, \texttt{FIND\_NODE}, \texttt{STORE\_PLAYER}, \texttt{STORE\_CHUNK}, \texttt{FIND\_PLAYER} and \texttt{FIND\_CHUNK}.
	
	In order to accommodate these new RPCs, the lookup procedure was made polymorphic, taking the appropriate RPC as an argument. Additionally, each node now has two separate storage tables, one for player data and one for chunk data. This implementation was preferable to the alternative of running two distinct \kademlia{} networks, as this would have significant additional overhead as two separate node states would need to be maintained for each node.
	
	\subsection{Generate Procedure}
	When a chunk is found not to exist in the network, it must be generated. In order to do this we need to first locate the server it should be generated on, then we need to check that server is up. Once we have confirmation that the server is running we send a request to the game server on that VoxelPopuli server to generate and initialise the node ready for players. If a node is down we move to the next nearest node. Once the chunk is generated we must store the $\langle$ IP address, port$\rangle$~of the node it was generated on in the network so that in future when we look up this chunk we will find the correct server. Algorithm~\ref{alg:generate} gives pseudocode of the generate procedure.
	\begin{algorithm}[!ht]
		\begin{algorithmic}
			\Function{Generate}{ChunkCoordinate}
				\State $key \gets \mathrm{sha1(ChunkCoordinates)}$
				\State $nodes \gets \mathrm{lookup(key)}$
				\For{$n \in nodes$}
					\State Send generate request to game server at $<n.ip, n.port+1>$ \Comment{Game server address is that of respective \kademlia{} node with port incremented.}
					\If{Successful}
						\State Call RPC \Call{STORE\_CHUNK}{$key, <n.ip, n.port+1>$}
						\State \Return Success
					\EndIf
				\EndFor
				\State \Return Failure
			\EndFunction
		\end{algorithmic}
		\caption{Generate Procedure Pseudocode.}
		\label{alg:generate}
	\end{algorithm}
	
	\section{Game Server}
	The game server is responsible for performing the computation and maintaining the state for a number of chunks concurrently. It is also responsible for providing clients with a method of querying the DHT to locate chunks in the \voxpop{} network and to retrieve player data.
	
	\subsection{Server State}
	The game server state consists of a set of active client connections (and two queues for each connection, one for receiving and one for sending data) and the state for each of the chunks it is responsible for. Chunk state consists of the world data for that chunk, represented as a 3D array\footnote{The current size of a chunk is $32\times32\times32$ blocks.},  of integers ranging between zero and three to represent different voxel types (see table~\ref{tab:blocks}); the set of players currently active in this chunk and their positions; a list of clients who are subscribed to updates on this chunk and $(x,y)$ -- the coordinates of this chunk in the world. Note here that a client may be connected to a single game server multiple times because it is connected to multiple of the chunks this server is responsible for.
	
	\begin{table}[!ht]
		\centering
		\begin{tabularx}{0.65\textwidth}{| l | l | l | l | X |}
			\hline
			\textbf{Block Type} & Air & Stone & Grass & Dirt \\
			\hline
			\textbf{Integer Value} & 0 & 1 & 2 & 3 \\
			\hline
		\end{tabularx}
		\caption{\voxpop{} block types.}
		\label{tab:blocks}
	\end{table}
	
	Chunks are either loaded or unloaded, unloaded chunks have no connected clients and their computation (i.e. game loop) is not currently being executed. A loaded chunk has a dedicated thread for performing the game computations of the chunk, a chunk is only loaded if it has a non-zero number of connected clients and is unloaded as soon as the last client disconnects.
	
	The game server's main thread runs continuously a loop it checks if there are any new connections, in which case it will perform the handshake procedure in \cref{sec:handshake}. Furthermore, it uses \texttt{select} to get a list of sockets ready for reading or writing. Then, for each readable socket it reads all data available and adds it to the socket's buffer, if it encounters a newline character (denoting the end of a packet) it puts the packet into the receive queue associated with this socket and clears the buffer. For each writeable socket it checks if there is queued data to send and if so, attempts to send it, removing the successfully sent data from the send queue associated with that socket.
	
	\subsection{Protocol}
	Here I outlines the protocol used to communicate between a \voxpop{} node and a client. It is separated into two distinct parts: the handshake procedure used to connect to a \voxpop{} node and determine the function of the connection and the game server protocol used to exchange game state between client and server each tick.
	\subsubsection{Handshake}
	\label{sec:handshake}
	When the server receives a new connection it waits for a JSON message from the new client, this message is decoded and it should contain a \emph{type} field. This \emph{type} field can take four different values, depending on which the server will take different actions. The possible values are:
	\begin{itemize}
		\item \textbf{``connect''} -- this means that the client wishes to connect to a chunk resident on this \voxpop{} server. There will be a field called \emph{chunk} which contains the chunk coordinates of the chunk to be connected to. The server will then perform the following:
		\begin{enumerate}
			\item Check if the relevant chunk is indeed resident and generated on this server. If not it will send a failure message to the client and return.
			\item Check whether the chunk is loaded and if not load it by starting a new chunk processing thread for this chunk (see \cref{sec:cthread}).
			\item Add this client to the chunk processing thread's client list.
		\end{enumerate}
		\item \textbf{``generate''} -- this means that the client (in fact another \voxpop{} server in this case) is requesting that a supplied chunk be generated. Again the \emph{chunk} field will be present, containing the coordinates of the chunk to be generated. The server will simply generate the chunk and add it to its set of chunks, however, it will not load it.
		\item \textbf{``dht''} -- this means that the client does not wish to use this node as a game server but instead as an access point to the DHT. In this case we launch a new, dedicated, thread to handle DHT queries, details of this thread can be found in \cref{sec:interface}.
		\item \textbf{``ping''} -- used to query whether a game server is alive, returns the \texttt{UTF-8} encoding of the string \emph{``pong''}.
	\end{itemize}
	Once the action has been completed successfully the server sends an acknowledgement to the client, informing it that the operation succeeded. Unless the connection is to be kept live (i.e. in the case of connecting to the game server (``connect'' packet) or DHT interfacing (``dht'' packet)) then the connection is closed.
	
	\subsubsection{Game}
	There are a number of messages which need to be exchanged between the client and game server. These allow the client(s) to inform the server that it has performed an action updating the world's state and allow the server to inform the client(s) of these, and other, state changes. Again these are exchanged as JSON messages of the format shown in table~\ref{tab:gameprotocol}.
	\begin{table}[!ht]
		\centering
		\begin{tabularx}{0.85\textwidth}{| l | l | X |}
			\hline
			Type (\texttt{int}) & Arguments (\texttt{list of floats}) & Player (\texttt{string})\\
			\hline
		\end{tabularx}
		\caption{Message format for the game server protocol.}
		\label{tab:gameprotocol}
	\end{table}

	The game server protocol consists of six types of packet which are exchanged between clients and servers. The details of each packet are in table~\ref{tab:gamepackets}. In order to signal the end of a packet each packet has a newline character appended to the end of its JSON representation. Packets are encoded as bytes using UTF-8 and then sent over a TCP socket between client and server.

	\begin{table}[!ht]
		\begin{tabularx}{\textwidth}{| l | l | X | p{2cm} |}
			\hline
			\bf Packet Type & \bf ID & \bf Purpose & \bf Arguments \\
			\hline
			\tt PLAYER\_REGISTER & $1$ & Used by clients to inform a server that a player is entering a particular chunk. Relayed to clients to inform them they should start rendering this player. & Player position $(x,y,z)$. \\
			\hline
			\tt PLAYER\_DEREGISTER & $2$ & Used by clients to inform a server that a player has left a particular chunk. Also triggers server to save player data to DHT. Relayed by server to clients to inform they should stop rendering this player. & None. \\
			\hline
			\tt PLAYER\_MOVE & $3$ & Used by clients to update inform the server of a change in a player's position. Relayed by server to clients so they can update where the player is being displayed. & New player position $(x,y,z,\theta)$, $\theta=$ yaw. \\
			\hline
			\tt CHUNK\_DATA & $5$ & Packs entire chunk block array into packet and sends to player so they can load and display a chunk. & Flattened chunk data array. \\
			\hline
			\tt TIME & $6$ & Informs the client that the in-game time has changed, sent every tick. & Current time as minutes from midnight (accurate to second). \\
			\hline
			\tt BLOCK\_CHANGE & $7$ & Used by a client to inform the server that the world has changed, server updates state and sends updated \texttt{CHUNK\_DATA} packet to all clients. & Position of block to change and new block type. \\
			\hline
		\end{tabularx}
		\caption{List of packets types exchanged between clients and game servers.}
		\label{tab:gamepackets}
	\end{table}
	
	\subsection{Chunk Thread}
	\label{sec:cthread}
	The chunk thread performs the computation for one chunk of the world. In theory, the network should contain no more than one chunk thread for any given chunk. Meaning that clients connecting to the same chunk will always find the same server.
	
	It is important here to make a distinction between a `client' and a `player', a client is any instance of the software described in \cref{sec:client} and has an associated player with a position in the world. This client may be connected to multiple chunk threads across the network, however, the player is only resident in one of these chunks, in all the other chunks the client is simply listening to updates from this server in order to render this chunk to the user.
	
	The computation runs in a continuous loop, it pops any packets from its receive queue and performs the appropriate action as explained in table~\ref{tab:gamepackets}, this may involve updating the game state and hence, sending packets to some of the clients connected. It then steps the game time and sends all players in this chunk a packet informing them of the time change. Next it checks if any player has not been heard of for a certain period of time and removes it, assuming it has gone offline.
	
	When there are no active clients for this chunk thread it will unload, saving the state and stopping the thread so as not to waste resources. This ensures that the number or chunk threads active on a given node at a particular time is kept to a minimum.
	
	\subsection{DHT Interface}
	\label{sec:interface}
	As previously explained, in order for clients to query the DHT (and thus retrieve their saved player data and the locations of chunks in the network) they must have some method to query the DHT. In order to do this they connect to a game server and perform the DHT handshake as described in \cref{sec:handshake}. This connection is then given a dedicated thread to respond to queries to the DHT. The current protocol allows a client to either get the $\langle$ IP, port$\rangle$~ for a particular chunk (the chunk is generated if does not exist in the network) or download player save data (position primarily).
	
	\subsubsection{Protocol}
	The thread process one request at a time and is not guaranteed to respond to requests issued while another request is pending. There are two packet types chunk query (type $0$) and player query (type $1$), a packet's first byte is its packet type. To query a chunk location a packet is sent consisting of a leading zero byte (packet type) followed by the JSON encoding of the coordinates of the chunk (as a JSON list). The server replies with the JSON list \texttt{[IP address, port]}. To query player data a packet consisting of a leading $1$ byte (packet type) followed by the JSON attribute-value pair: \texttt{\{`name':player\_name\}} where \texttt{player\_name} is the username of the player to query. The server responds with the JSON representation of the player's coordinates. If the player was not found then it returns $(0,32,0)$ (the default spawn position). Note that all the JSON strings are encoded as bytes for sending using UTF-8.
	
	\section{Client}
	\label{sec:client}
	The client is the user's entry point into the world, it renders the world (and other players -- rendered as grey cuboids) and allows the player to move around within it; it provides a basic physics simulation for realistic movement and collisions and it allows the user to make changes to the world by breaking and placing blocks. A screenshot of the client in action can be seen in fig~\ref{fig:client}. The client was written in C\# using the \unity{} game engine.
	
	\begin{figure}[!ht]
		\includegraphics[width=\textwidth]{client.png}
		\caption{A screenshot of the \voxpop{} client connected to a world.}
		\label{fig:client}
	\end{figure}
	
	\subsection{Architecture} % talk about multithreading and packets -> updates
	The client operates two main threads as well as a two threads (one each for receiving and sending) for each chunk it is connected to. The two main threads exist to separate long, blocking processes from the thread rendering the world, this prevents frame rate stuttering\footnote{Where the frame rate drops briefly, causing the game to appear to momentarily freeze}. The two main threads will be referred to as the \emph{network thread} and the \emph{game thread}.
	
	\begin{table}[!ht]
		\begin{tabularx}{\textwidth}{| l | X | p{3cm} |}
			\hline
			\bf Update Type & Purpose & Arguments \\
			\hline
			\tt PLAYER\_MOVE & Used by the network thread to inform the game thread that another player has moved and the object representing that player should be moved. & Player name and new position.\\
			\cline{2-2}
			\tt & Used by the game thread to inform the network thread that the player associated with this client has moved so that the network thread can relay this to the server. & \\
			\hline
			\tt LOAD\_CHUNK & Used by the network thread to make the game thread load in newly downloaded chunk data. & Chunk coordinates, block array and mesh data. (Encapsulated as a `Chunk` object.) \\
			\hline
			\tt UNLOAD\_CHUNK & Used by the network thread to make the game thread unload a chunk, removing it from the game world. & Chunk coordinates. \\
			\hline
			\tt PLAYER\_ADD & Used by the network thread to inform the game thread of a player appearing in a particular chunk, the game thread will add an object to the world to represent this player. & New player's position and name. \\
			\hline
			\tt PLAYER\_REMOVE & Used by the network thread to inform the game thread a player has left a particular chunk, the game thread removes the player from the world. & The player's name. \\
			\hline
			\tt TIME & Network thread informs the game thread a \texttt{TIME} packet has been received, the game thread will move the game's light object to reflect this. & The new time. \\
			\hline
			\tt BLOCK\_CHANGE & Used by the game thread to inform the network thread the player has broken or placed a block. & The position of the block to be updated and its new type.\\
			\hline
		\end{tabularx}
		\caption{Details of the updates exchanged between the network and game threads of the client.}
		\label{tab:updates}
	\end{table}
	
	\subsubsection{Network Thread}
	\label{sec:nthread}
	The network thread has access to send/receive queues for each chunk the client is connected to. It takes received packets from these queues and processes them ready to be sent to the game thread. Some packets require an amount of computationally intensive pre-processing before they are ready to be used by the game. For example, packets of type \texttt{CHUNK\_DATA} require a large amount of processing as the array needs to be unflattened and then the chunk's mesh data needs to be generated -- this can take $10$s of frames. This computation is performed in the network thread which has no bearing on the client's frame rate. Once any pre-processing is performed a state update is sent to the game thread which is rendering the world.
	
	The network thread also receives updates from the game thread, these updates can be player movement or block changes. Where appropriate, it converts these updates into a packet and queues these for sending to the relevant chunk's server.
	
	Finally, the network thread ensures that the correct chunks are loaded, locating and loading those which are needed and unloading those which are no longer needed. This is discussed in detail in \cref{sec:loading}. To do this is maintains a connection to a \voxpop{} node where the connection has been setup to be in DHT query mode, allowing the client to use this node to query the DHT.
	
	\subsubsection{Game Thread}
	\label{sec:gamethread}
	This is the main \unity{} thread. The part that is of interest to this project is the \texttt{update()} method in our \texttt{World} script which is executed each frame. This has access to the incoming and outgoing update queues used to exchange state updates between the network thread and the game thread. The game thread simply performs the necessary steps to make state updates received from the network thread visible to the user, and it sends updates to the network thread when the user does something which updates the state of the world.
	
	Details of the types of updates exchanged can be seen in table~\ref{tab:updates}. Some of these are very similar to the packets from table~\ref{tab:gamepackets} as expected. Within the game thread each player has an object to represent it (in this prototype they are simply a cube) and each chunk has an object to represent it (a 3D mesh with an associated `Chunk' object containing the block array and the mesh's data -- details of mesh data generation are in \cref{sec:mesh}).
	
	\subsection{Chunk Mesh Generation}
	\label{sec:mesh}
	In computer graphics, a \emph{mesh} is a collection of vertices, edges and faces which make up a 3D object. For each chunk we must generate a mesh from the block array in order to render the chunk to the player. In order to do this we must construct two arrays; firstly, we need to compute the array of vertices in the mesh, then we need the array mapping these vertices to triangles. In order to do this, for each triangle we need the indices of the three vertices that make up that triangle, we then add these three indices to the triangles array.
	
	\begin{figure}[!ht]
		\centering
		\includegraphics[width=0.5\textwidth]{diag2.png}
		\caption{A diagram of mapping between vertices and triangles for a simplified face of a cube.}
		\label{fig:mesh}
	\end{figure}
	
	In order to apply textures to the faces we also need to provide a texture coordinate (often referred to as a \emph{uv} coordinate) to each vertex. This is supplied in an array too, where the indices in the vertex array correspond to the indices of the texture coordinate of that vertex in the \emph{uv} array. Once we have computed our vertex, triangle and \emph{uv} arrays we can supply these to \unity{} which will be able to generate and render the mesh for us. This process is very computationally intensive, for our chunks it involves iterating over all $32^3 = 32768$ blocks and performing six iterations (one per face) for each block. Pseudocode for the mesh generation algorithm is supplied in algorithm~\ref{alg:mesh}, this approach is based on similar methods found in various open source voxel-based games. Figure~\ref{fig:mesh} shows how a face of the cube is produced by two triangles.
	
	\begin{algorithm}[!ht]
		\begin{algorithmic}
			\State blocks[][][] $\gets$ chunk block array;
			\State vertices[] $\gets$ new vec3 array;
			\State triangles[] $\gets$ new int array;
			\State uvs[] $\gets$ new vec3 array;
			\State vertex $\gets 0$;
			\For{$0 \leq x <$ \tt CHUNK\_SIZE}
				\For{$0 \leq y <$ \tt CHUNK\_SIZE}
					\For{$0 \leq z <$ \tt CHUNK\_SIZE}
						\For{each face}
							\If{blocks[x][y][z] is air OR\\ blocks[x][y][z] + faceNormal is not air}
								\State continue;
							\EndIf
							\State add vertices for this face;
							\State add \emph{uv} coordinates for correct texture;
							\State triangles += [vertex+0, vertex+1, vertex+2];
							\State triangles += [vertex+2, vertex+1, vertex+3];
							\State vertex $\gets$ vertex + 4;
						\EndFor
					\EndFor
				\EndFor
			\EndFor
		\end{algorithmic}
		\caption{The mesh generation algorithm used by the client.}
		\label{alg:mesh}
	\end{algorithm}
	
	\subsection{Chunk Loading \& Unloading}
	\label{sec:loading}
	As noted in \cref{sec:nthread} the network thread is responsible for ensuring the correct chunks are loaded, it uses the DHT query connection for this purpose. The rule for determining whether a chunk should be loaded is simple: a chunk should be loaded if it is not already loaded and it is within a three by three grid of chunks centred on the player. Similarly, a chunk should be unloaded if it is outside of a nine by nine grid of chunks centred on the player.
	
	Each time the network thread processes a \texttt{PLAYER\_MOVE} update from the game thread it checks to see if there are any chunks to unload, if so it unloads them (i.e. ends the connection and sends a \texttt{CHUNK\_UNLOAD} update to the game thread). Then it checks if there are any chunks to load, if so it queries their locations using the DHT connection, connects to them and loads them (it only sends the \texttt{CHUNK\_LOAD} update when it receives the \texttt{CHUNK\_DATA} packet from the server for this chunk). This process is outlined in algorithm~\ref{alg:load}.
	
	\begin{algorithm}[!ht]
		\begin{algorithmic}
			\For{chunk $\in$ loaded}
				\If{$|\mathrm{chunk.x} - \mathrm{player.chunk.x}| \geq 5$ OR $|\mathrm{chunk.y} - \mathrm{player.chunk.y}| \geq 5$}
					\State unload(chunk);
				\EndIf
			\EndFor
			\State chunk\_position $\gets$ (player.chunk.x, player.chunk.y);
			\For{$-1 \leq i \leq 1$}
				\For{$-1 \leq j \leq 1$}
					\If{chunk at chunk\_position + $(i,j)$ is not loaded}
						\State s $\gets$ query\_chunk\_from\_dht(chunk\_position + $(i,j)$);
						\State load chunk from server s;
					\EndIf
				\EndFor
			\EndFor
		\end{algorithmic}
		\caption{Algorithm for maintaining the correct set of loaded chunks by the client.}
		\label{alg:load}
	\end{algorithm}
	
	\section{Overview}
	This section provides an overview of the project and the files contained within my repository (which are shown in figure~\ref{fig:dir} shows).
	\begin{wrapfigure}[29]{R}{6.5cm}
		\dirtree{%
			.1 \voxpop\DTcomment{1860 lines}.
			.2 Client\DTcomment{819 lines}.
			.3 Assets.
			.4 Code.
			.5 Chunk.cs.
			.5 Constants.cs.
			.5 Controller.cs.
			.5 Network.cs.
			.5 WireFrame.cs.
			.5 World.cs.
			.4 $\dots$.
			.2 Server\DTcomment{853 lines}.
			.3 main.py\DTcomment{153 lines}.
			.3 game\DTcomment{186 lines.}.
			.4 chunkthread.py.
			.4 const.py.
			.4 world.py.
			.3 kademlia\DTcomment{401 lines}.
			.4 node.py.
			.4 protocol.py.
			.4 router.py.
			.4 server.py.
			.4 storage.py.
			.3 tests\DTcomment{113 lines}.
			.4 $\dots$.
			.2 Test Client\DTcomment{188 lines}.
			.3 test\_client.py.
			.3 test\_many\_chunk.py.
			.3 test\_single\_chunk.py.
		}
		\caption{Directory overview of \voxpop{}.}
		\label{fig:dir}
	\end{wrapfigure}
	
	\subsection{Client}
	The client takes the form of a \unity{} project. As noted, Unity allows you to assign scripts to game objects which then have methods such as \texttt{Update()} and \texttt{Start()} which are called at the appropriate times (in these cases: each frame and when the object is created respectively).
	
	The client has six files of code, three of which contains scripts able to be assigned to objects. {\tt Controller.cs} is a script assigned to the player object which takes user input and moves the player, obeying physics and collisions. {\tt WireFrame.cs} is another script applied to the player object which simply renders a wire frame cube around the block the player is looking at. {\tt World.cs} is a script applied to the (initially) world object which performs the game thread actions described in \cref{sec:gamethread}.
	
	The other three files contain classes and data used by the aforementioned scripts. In {\tt Chunk.cs} we have a class for chunks, these contain the chunk's block array and mesh data, and provide a number of methods for querying the contents of the chunk (such as {\tt IsSolid(Vector3 localPos)} for querying if a block is solid at position {\tt localPos} relative to the chunk). {\tt Network.cs} contains the {\tt NetworkThread} class which is used to execute the network thread as explained in \cref{sec:nthread}. Finally, {\tt Constants.cs} contains a number of constants such as the tick rate, chunk size and various pieces of data needed for mesh generation.
	
	\subsection{Server}
	The server is split into two main parts, the \kademlia{} implementation and the game server. Within the game server we have {\tt chunkthread.py} which contains the {\tt ChunkThread} class which is used to execute the chunk thread as discussed in \cref{sec:cthread}. We also have {\tt world.py} which contains classes for maintaining state about the world, namely {\tt Chunk} and {\tt Player}. We have {\tt const.py} which contains a number of important constants such as a {\tt PacketType} enumeration. Finally we have {\tt main.py} which performs the main loop of a \voxpop{} node, i.e. waiting for new connections and performing the handshake procedure from \cref{sec:handshake}. The code for the DHT interface thread (see \cref{sec:interface}) is here too.
	
	Within the \kademlia{} folder we have {\tt node.py} and {\tt storage.py} which contain simple classes to represent a \kademlia{} Node and the local storage on a node respectively. {\tt router.py} contains the \K-bucket class and routing table class. {\tt protocol.py} contains the \kademlia{} RPC implementation as well as the RPC framework outlined in \cref{sec:rpcframework}. Finally, {\tt server.py} contains the high level interface for the \kademlia{} network, allowing the following operations to be performed: {\tt run()} (initialise a node), {\tt get\_chunk()} (lookup a chunk and generate it if it does not exist), {\tt get\_player()} and {\tt save\_player()} (retrieve or save player save data), {\tt generate\_chunk()} and {\tt republish\_chunk()}.
	
	\subsection{Test Client}
	The test client is a suite of scripst which simply initialises a configurable number of dummy players and connects them to a single server or to a network in random or configurable positions, it can then move them around randomly to simulate activity.
	
	There are three scripts, two of which connect to a single server and one which connects to the network as a whole. First we have {\tt test\_sing\_chunk.py} which connects to a single server and loads a number of players into a single chunk and then moves them around within the chunk every tick ($20$ times a second). We also have {\tt test\_many\_chunks.py} which connects to a single server and loads a number of players into a configurable number of chunks on that server, the players are evenly distributed between chunks and then moved randomly as above.
	
	Finally we have {\tt test\_client.py} which connects to a server to query the DHT. It then spawns a number of dummy players who connect to the network and move around the world, moving between chunks this time and connecting to the appropriate server by querying the DHT -- this client most accurately simulates real players and is used in my large scale test of the system.
		
	\clearpage
	\chapter{Evaluation}
	\voxpop{} meets the original success criteria and further goals set out in \cref{sec:req}. It has a number of limitations which are explored in \cref{sec:limits}.
	
	\section{Methodology}
	This project aims to explore the viability of using a peer-to-peer network topology for a large online world with potentially thousands of players. Unfortunately I do not have access to enough machines (and indeed players) to accurately field test the project. I have instead performed two main types of testing to prove the viability of \voxpop{} as a MMO engine. I have performed local simulations using simulated players (created by my test client) to demonstrate the load that a single server experiences under different scenarios. Then I created a small network of around $100$ nodes which was populated by simulated players to prove that the system is able to perform at scale.
	
	I also tested the system by subjecting it to failure modes such as node failure, demonstrating that it continues to operate correctly in these scenarios. I also investigated the security of \voxpop{} and its susceptibility to various attacks. Additionally, I investigated the adherence of the \kademlia{} implementation to the specification and performed some simple tests on the client.
	
	Finally, I explored the limitations that the peer-to-peer approach presents when compared to the more commonplace client-server model.
	
	\section{\kademlia{} Implementation}
	In order to prove that my \kademlia{} implementation conforms with the specification, as required by my success criteria, I performed two types of testing.
	
	\subsection{Unit Testing}
	Unit testing was performed on core components of the \kademlia{} specification. These were informed by the unit tests for \emph{OpenDHT}~\cite{opendht} and the Python \kademlia{} library~\cite{pykademlia}. Components tested with unit tests include the \K-bucket class -- ensuring that it performs correctly as nodes are added and removed, the routing table -- ensuring contacts are processed correctly and that it responds correctly to queries and the storage implementation.
	
	\subsection{RPC Testing}
	\subsubsection{Setup}
	The \kademlia{} specification sets out requirements for the time complexity of each RPC, I needed to prove that my implementation adheres to these requirements. In order to do this I setup a single dedicated server and instantiated a number of \kademlia{} nodes and bootstrapped them to a network to give me a network of a known size. I then performed the relevant RPC a number of times and recorded the average time and standard deviation. The plot can be seen in figure~\ref{fig:kadperforma}.
	\begin{figure}[!ht]
		\includegraphics[width=\textwidth]{rpcperformance.png}
		\caption[RPC Performance]{Computation time of each \kademlia{} RPC as network size increases. The \kademlia{} specification states these should all be $\mathcal{O}(1)$. Error bars represent $\pm\sigma$ (standard deviation) over $100$ trials.}
		\label{fig:kadperforma}
	\end{figure}

	\subsubsection{Analysis}
	As can be seen each of {\tt PING}, {\tt STORE\_CHUNK} and {\tt STORE\_PLAYER} are clearly $\mathcal{O}(1)$. The plots for {\tt FIND\_NODE}, {\tt FIND\_CHUNK} and {\tt FIND\_PLAYER} exhibit a slightly different shape. These RPCs still operate in  $\mathcal{O}(1)$ time, however, because the find operations must sort the nodes it knowns of, this takes $\mathcal{O}(n\log(n))$ time, however, as the number of nodes in the routing table is capped at $160 \cdot \K{}$ ($3200$ in the common case where $\K = 20$ -- used in my implementation) this is still $\mathcal{O}(1)$ as the number of nodes grows large. You can see the graph is beginning to plateau at for higher $n$.
	\begin{figure}[!ht]
		\includegraphics[width=\textwidth]{lookupgraph.png}
		\caption[Lookup Procedure Performance]{Number of nodes contacted by a {\tt lookup} procedure as network size increases. The \kademlia{} specification states that this should grow at a rate of $\mathcal{O}(\log{n})$, where $n$ is number of nodes. Error bars represent $\pm\sigma$ (standard deviation) over $100$ trials.}
		\label{fig:kadperformb}
	\end{figure}
	\subsection{Lookup Procedure}
	The \kademlia{} specification clearly states that the lookup procedure must contact $\mathcal{O}(\log(n))$ nodes. I performed a test where I recorded each node that was contacted during a lookup for increasing network sizes. Figure~\ref{fig:kadperformb} shows the results. As can be clearly seen, the growth fits a logarithmic curve perfectly, demonstrating that my implementation correctly adheres to this part of the standard.
	\section{Scalability}
	It is important that we prove the system scales to a large number of players. In order to do this without needing to purchase a large number of servers I performed several tests on a single instance. This was to test the performance under several different types of player load to verify that a single server can cope with a representative number of players.
	
	I then created a 100 node network hosted on my dedicated server and tested it with 1000 simulated players to demonstrate the system functions correctly when deployed as a network in a large system. In this instance communication will have been slightly slower than usual due to communication being handled locally on the server, but this is acceptable as it still gives useful a lower bound on performance. I would note that to make this function correctly I had to increase the UDP and TCP buffer sizes on my server due to the high volume of traffic being exchanged internally between nodes.
	\subsection{Local Simulation}
	\label{sec:localsim}
	\subsubsection{Setup}
	I performed 4 local simulations on a single node, each was run for 1 hour with variable player loads. I ran a single \voxpop{} node on my dedicated server and connected one of my test client scripts to it to simulate player activity. The plots in figure~\ref{fig:plots} show the CPU and memory usage under various scenarios.
	\begin{figure}[!ht]
		\begin{subfigure}{0.5\textwidth}
			\includegraphics[width=\textwidth]{plot1.png}
			\caption{Performance (50 players/1 chunk)}
			\label{fig:plot1}
		\end{subfigure}
		\begin{subfigure}{0.5\textwidth}
			\includegraphics[width=\textwidth]{plot2.png}
			\caption{Performance (50 players/50 chunks)}
			\label{fig:plot2}
		\end{subfigure}
		\begin{subfigure}{0.5\textwidth}
			\includegraphics[width=\textwidth]{plot3.png}
			\caption{Performance (50 players/10 chunks)}
			\label{fig:plot3}
		\end{subfigure}
		\begin{subfigure}{0.5\textwidth}
			\includegraphics[width=\textwidth]{plot4.png}
			\caption{Performance (100 players/10 chunks)}
			\label{fig:plot4}
		\end{subfigure}
		\caption{Plots showing CPU utilization and memory usage of \voxpop{} nodes under different scenarios.}
		\label{fig:plots}
	\end{figure}
	
	\subsubsection{Analysis}
	As can be seen from figure~\ref{fig:plot1} the single thread performance is reaching $90\%$ utilization at about $50$ players in a single chunk. This is fairly typical of online games and it is at around this point that \emph{Minecraft} servers begin to strain. This scenario is unlikely to be realised frequently in a deployed network running an MMO as this number of players tend not to occupy the same area concurrently.
	
	In figure~\ref{fig:plot2} we have $50$ clients connected to $50$ separate chunks on the same server. This performs better as it is making use of both CPU cores on my dedicated server. In line with this expectation CPU utilisation is approximately half that of the single chunk case. This also represents an unrealistic scenario as we are unlikely to have players evenly distributed across the world so evenly, there will be likely more clustering. This test does demonstrate, however, that many chunk threads can operate concurrently without incurring a significant performance penalty.
	
	In figure~\ref{fig:plot3} we instantiate $50$ players and evenly distribute them over ten chunks, this is a more realistic scenario and is designed to represent players playing with friends in small groups occupying the same areas. Performance here is improved over the 50 chunk case as expected, likely due to a reduction in the context-switching overhead. Averaging at around $25\%$ utilisation the system is not struggling with this scenario, suggesting more players could be supported in a similar scenario. In order to test this I performed another test by doubling the number of players while keeping the number of chunks constant, in figure~\ref{fig:plot4} we have $100$ players across ten chunks, and CPU utilisation averages around $55\%$. Suggesting a linear relationship between number of players and performance when the number of chunks is kept constant.
	
	To verify the linearity I plotted CPU utilisation as player count was increased from one to fifty. I wrote a simple script to connect a new client every two seconds. The plot it shown in figure~\ref{fig:linear50} and follows a roughly linear shape. This supports my theory that performance with constant number of chunks is proportional to the number of players connected.
	\begin{figure}[!ht]
		\includegraphics[width=\textwidth]{linear.png}
		\caption{Performance as player count increases ($0-50$).}
		\label{fig:linear50}
	\end{figure}

	I did a further test using the same script, instead adding a node every second and going from one to one-hundred. The resulting plot is shown in figure~\ref{fig:linear100}. The linear relationship persists up to around seventy nodes. However, we see CPU usage then plateaus at $100\%$ and memory usage starts to rise, this is indicative that the system is struggling to handle both the garbage collector and the game server computation. Thus as a lower bound we can suggest that approximately seventy players can exist on a single chunk thread before performance issues begin to arise.
	\begin{figure}[!ht]
		\includegraphics[width=\textwidth]{linear100.png}
		\caption{Performance as player count increases ($0-100$).}
		\label{fig:linear100}
	\end{figure}

	Throughout this analysis the memory usage has remained approximately constant throughout the simulation, this is because memory usage is proportional to the number of chunks active and in each simulation this was constant. The memory overhead for a single chunk is below ten megabytes and consists only of the chunk data ($\sim260$KB) and player data ($<1$KB per player) so this is not of major concern here. In a more complex system where there are other entities the server must keep track of such as monsters this may become a significant concern as it is for \emph{Minecraft} servers, which can typically need hundreds of gigabytes of memory to function correctly.
	
	\subsection{Large Scale Simulation}
	\subsubsection{Setup}
	To demonstrate \voxpop{} functioning at scale I setup a $100$ node network on my dedicated server and connected $1000$ dummy players using {\tt test\_client.py} from my test client scripts. As some servers would have high load and others would have very low load depending on which chunks are loaded at any given time I found it more useful to plot the maximum CPU at any given time, as the average was very low (around half a percent). The plot is shown in figure~\ref{fig:big}.
	\begin{figure}[!ht]
		\includegraphics[width=\textwidth]{100node.png}
		\caption{Performance of a $100$ node network with $1000$ players connected.}
		\label{fig:big}
	\end{figure}

	\subsubsection{Analysis}
	Initially, the utilisation is not constant as nodes are still joining the network (it takes approximately ten minutes for $1000$ nodes to join using the test client). Then we go into normal operation and the maximum load stays approximately at $10\%$, this is in line with our analysis in \cref{sec:localsim} as there would be approximately ten nodes per server on average. The system was left running for 1 hour and experienced no problems, I was able to connect with my graphical client and explore the world, viewing the dummy players teleporting around randomly. This can be seen in figure~\ref{fig:dummies}.
	\begin{figure}[!ht]
		\includegraphics[width=\textwidth]{dummies.png}
		\caption{Screenshot of dummy players in the \voxpop{} world.}
		\label{fig:dummies}
	\end{figure}

	\section{Client}
	The client performs as required, the player is able to explore and interact with the world and the client is performant -- running consistently at $60$FPS+ (frames per second) on both my dedicated GPU (Nvidia GeForce GTX $1650$ Max-Q) and my integrated graphics (Intel Iris Plus Graphics).

	\section{Node Failure}
	The system performs correctly in the face of node failure. When the client loses connection to a node it queries the node it is using for DHT queries. This server finds the required node and discovers that it is not online. It thus performs the generate procedure (see algorithm~\ref{alg:generate}) before returning the new node to the client. This takes at most a couple of seconds and produces minimal disruption to gameplay.
	
	\section{Current Limitations}
	\label{sec:limits}
	\subsection{Scale}
	I will now attempt to reason whether \voxpop{} can handle a world the size of a typical MMO with a player count to match. As a baseline for player count I will be using the MMO \emph{EVE Online} which had a peak concurrent player count $63,160$~\cite{eve} in 2011. The EVE world is sparse so I will use the \emph{World of Warcraft} world size of approximately $200km^2$, estimates vary so I took the highest I could find~\cite{wowsize}. Assuming each voxel in my game is $1m^3$, each chunk covers $1024m^2 \approx 0.001km^2$. Thus we would need $200,000$ chunks to represent this world. Assuming each server controls one hundred chunks we would need a network of $2000$ nodes to store this world. This is well within the capacity of the \kademlia{} network as all RPCs have $\mathcal{O}(1)$ cost and the lookup operation grows slow ($\mathcal{O}(\log(n))$).
	
	Assuming that players are evenly distributed across the world that would put $32$ players in each chunk, this is obviously going to overload the server. However, there are likely to be regions that are not currently loaded and regions which are more popular. So if we generously assume that $50\%$ of regions are unloaded, then we have $64$ players per loaded chunk and only $50$ chunks loaded per server. This is still likely to exceed the performance limitations of the mid-range machine I used in my testing.
	
	Alternatively, assume that the network is of size $20,000$, then we have ten chunks per node and still $32$ players per chunk, this is more feasible. This approach might be realistic, consider that the peak active player count (this is not concurrent, but playing regularly) of \emph{World of Warcraft} was twelve million~\cite{wowbig}. Only $0.17\%$ of players would need to setup a server for the system to be viable. However, there is no escaping that this is a clear limitation of my approach. The fact that each chunk operates its own thread is clearly debilitating to the performance of the system. An approach used by \emph{SpatialOS}~\cite{SpatialOS} is to have chunks grouped together spatially and computations performed on the entire region of space in one thread -- this is an approach I would like to explore in my system in the future as I think it could drastically improve scalability.
	
	\subsection{Security}
	\label{sec:security}
	\kademlia{} is resistant to certain DoS attacks, the preference for old contacts ensures that flooding the system with new nodes will not cause existing nodes' state to be flushed, giving control of the network to the attacker. However, \kademlia{} remains vulnerable to various types of attack. Including but not limited to eclipse and Sybil attacks. Additionally, \kademlia{} does not prevent nodes from impersonating others when bootstrapping, creating ambiguity about who is the \emph{true} owner of a particular ID. Ultimately, a determined attacker would be able to detrimentally influence the \voxpop{} network should they so desire -- and in an MMO context, where there would be motivation to cheat, this is almost guaranteed to happen.
	
	Furthermore on this point, at present the game server trusts all requests coming from the player and does not attempt to verify the validity of their movements. This would be simple to implement -- store the player's state in the network and verify the what the player says is consistent with the stored state (e.g. don't allow player to move across many chunks in one movement but do allow them to move to an adjacent chunk).
	
	\subsection{Data Loss}
	One significant issue with the current implementation is that when a node dies all the chunks stored within it are lost and will be regenerated by another node when they are needed. This means that anything a player has constructed within that chunk is lost. An approach I proposed as a potential extension to this project could be used to mitigate this problem. I propose that instead of having a single node responsible for each chunk, have a node and several backup nodes, periodically the main node will multicast its state to the backup nodes, should this node die, the backup nodes elect a successor and then induct a new node into the backup set. This could be done simply using the $n$ nearest nodes to the chunk's key.
	
	\clearpage
	\chapter{Conclusion}
	\voxpop{} set out to investigate the feasibility of creating a scalable, peer-to-peer distributed online world. The project aimed to explore the peer-to-peer approach as it differs from the more commonly used centralised, client-server model. Ultimately my project was a success, it has fulfilled the success criteria and thorough testing has revealed it to be scalable to the size of a typical MMO (estimated to be a maximum of$\sim200,000$ chunks and $\sim60,000$ players) when given a sufficiently large network, though the proposed architectural tweaks could improve this.
	
	One issue with peer-to-peer systems is security, and \voxpop{} is no exception. The system suffers from vulnerability to various types of attacks as noted previously. In a commercial environment with a real MMO we would wish to mitigate this to ensure fairness of the game and robustness of the service. Thus we might require some notion of trusted, or authoritative meta-servers perhaps controlled by the developers.
	
	Having authoritative, centralised servers goes against the spirit of \voxpop{}. As such, I would propose that in the event of the MMO becoming unprofitable that the developers could allow the game's community to host their own authoritative servers. Alternatively, one could implement some sort of consensus system to improve security, this would not be perfect but would be a significant improvement and would fit more with the principles of the project. A distributed ledger could be employed to solve this problem in a way that fits with the open and peer-to-peer philosophy of \voxpop{}.
	
	Additionally, to ensure fairness and prevent cheating, the changes required to prevent players from fabricating their in-game state discussed in \cref{sec:security} would need to be implemented. This would prevent players from being able to easily create hacked clients; note that this is a common problem with online games and not an issue with the peer-to-peer approach.
	
	As noted in the evaluation, the system requires a large number of nodes in order to scale to typical MMO sizes. One possible solution is to cluster chunks spatially. At present many spatially disparate chunks may be processed by the same server, this means that a single server may need in excess of $100$ threads performing computationally heavy processes, this is obviously not feasible except on extremely powerful machines. By clustering chunks spatially, we can reduce the computational requirement by simulating contiguous clusters of chunks together. This reduces the number of threads, meaning that while a single server may be responsible for hundreds of chunks, it may need only a few threads, one for each cluster of spatially contiguous chunks.
	
	In a future version of this project I would like to include implementation of a distributed consensus system to improve security of the system. In particularly, I would like to explore the possibility of using a distributed ledger to achieve this. This would add significantly to the size of the implementation but would be necessary for the desired use case. Furthermore, I would like to include my suggested spatial clustering improvement in order to maximise performance, this would require a small additional programming effort and I estimate would significantly improve performance.
	
	%TC:ignore
	\clearpage
	
	\addcontentsline{toc}{chapter}{Bibliography}
	\begin{thebibliography}{9}
		\bibitem{kademlia} Maymounkov, P. and Mazières, D. Kademlia: A Peer-to-peer Information System Based on the XOR Metric. \url{https://pdos.csail.mit.edu/~petar/papers/maymounkov-kademlia-lncs.pdf}. Accessed: 2019-10-16.
		\bibitem{shard} ``Sharding'' on Wikipedia. \url{https://en.wikipedia.org/wiki/Shard_(database_architecture)}. Accessed: 2019-10-15.
		\bibitem{clubpenguin} ``Club Penguin is shutting down'' -- TechCrunch. \url{https://techcrunch.com/2017/01/31/club-penguin-is-shutting-down/}. Accessed: 2019-10-15.
		\bibitem{SpatialOS} SpatialOS by Improbable. \url{https://improbable.io/spatialos}. Accessed: 2020-03-20.
		\bibitem{pykademlia} \emph{Kademlia} Python Library. \url{https://github.com/bmuller/kademlia/}. Accessed: 2020-03-20.
		\bibitem{source} \emph{Source} Engine Multiplayer Networking, Valve. \url{https://developer.valvesoftware.com/wiki/Source_Multiplayer_Networking}. Accessed: 25-3-2020.
		\bibitem{perlin} ``Perlin Noise'' on Wikipedia. \url{https://en.wikipedia.org/wiki/Perlin_noise}. Accessed: 2019-10-17.
		\bibitem{simplex} ``Simplex Noise'' on Wikipedia.
		\url{https://en.wikipedia.org/wiki/Simplex_noise}. Accessed: 2019-03-27.
		\bibitem{opendht} \emph{OpenDHT} on GitHub. \url{https://github.com/savoirfairelinux/opendht}. Accessed: 2020-04-16.
		\bibitem{eve} ``Eve Online has over 360,000 players. 63,170 simultaneous users in January'' -- PC Gamer. \url{https://www.pcgamer.com/eve-online-has-over-360000-players-63170-simultaneous-users-in-january/}. Accessed: 2020-04-17.
		\bibitem{wowsize} Estimation of World of Warcraft world size. \url{http://tobolds.blogspot.com/2007/01/how-big-is-azeroth.html}. Accessed: 2020-04-17.
		\bibitem{wowbig} ``World of Warcraft Classic: Hit game goes back to basics
		'' -- BBC News. \url{https://www.bbc.co.uk/news/technology-49448935}. Accessed: 2020-04-17.
		\bibitem{verge} ``This Minecraft Library is Making Censored Journalism Accessible All Over the World''. \url{https://www.theverge.com/2020/3/18/21184041/minecraft-library-censored-journalism-reporters-without-borders}. Accessed: 2020-05-02.
	\end{thebibliography}
	\clearpage
	
	\appendix
	\chapter{Proposal}
	\documentclass[12pt,a4paper]{article}

\usepackage[hidelinks]{hyperref}

\begin{document}
	\title{Project Proposal -- A Decentralised Peer-to-Peer World}
	\author{Samuel J. Sully}
	\date{14 October 2019}
	\maketitle
	\thispagestyle{empty}
	
	\noindent
	\textbf{Project Supervisor:} Prof. Jon Crowcroft\\
	\textbf{Director of Studies:} Prof. Alan Mycroft\\
	\textbf{Project Overseers:} Prof. Marcelo Fiore \&  Dr. Amanda Prorok

	\section{Introduction}
	
	The Massively Multiplayer Online game genre is popular in modern gaming, often involving thousands of concurrent players~\cite{wowstat}. These games are typically implemented using a client-server model, requiring some form of `sharding'~\cite{shard} whereby players are separated into independent instances of the same world meaning that players can only interact with others connected to the same instance (shard) as them.
	
	An alternative approach is to distribute the game world over a peer-to-peer network, where each peer (or group of peers) is responsible for managing a small part of the game world. This helps with load balancing, while ensuring that all players are -- effectively -- in the same virtual world. It also has the advantage of decentralising the game world, increasing its resilience to some failure modes.
	
	This has the advantage that you are able to have a user make persistent changes to the game world, in a sharding implementation as the world is replicated it is hard to have a user make a meaningful change to the world, preventing Minecraft (or similar) like worlds which are fully customisable.
	
	Additionally, when large scale MMOs cease to be profitable for the developers, who operate the servers, they will often shut down the servers~\cite{down} even if there remains a large active player base. By allowing individuals to setup their own game servers to help run the MMO as part of the P2P network, we ensure that the game can survive, at no cost to the developers, as long as there is a community dedicated to ensuring its survival. This is typically not possible for MMOs, only for games which have a client-server multiplayer system.
	
	One issue with this approach is that, if a rogue server operator decides to behave maliciously, they could modify their server to -- for example -- give themselves an unfair advantage when in territory managed by their server. To combat this one could include a consensus/voting system~\cite{peerreview}.
	
	\section{Starting Point}
	
	Creation of the distributed world will rely mostly on two Part IB course:
	\begin{itemize}
		\item Concurrent and Distributed Systems
		\item Computer Networking
	\end{itemize}
	as well as drawing on material from the Part II course Principles of Communication.
	
	The Concurrent and Distributed Systems course will be useful as this project is an example of a distributed system. I will be making use of Remote Procedure Calls (RPCs) in my project which are covered in this course, one of my proposed extensions aims to include a consensus or voting system -- these topics are covered extensively in the CDS course.
	
	The Computer Networking course will be relevant as it covers Distributed Hash Tables (DHTs) which I intend to use as the fundamental basis for my project. Other concepts introduced in this course, and further developed in the Part II course Principles of Communications (such as routing) will also be relevant to my project.
	
	Additionally, the Part IA course Introduction to Graphics and the Part IB course Further Graphics will be useful in the development of a simple client to connect to and interact with my distributed world, this client would likely be written using OpenGL which is covered extensively in these two courses. It will, however, be very primitive indeed as it will be only a prototype.
	
	\section{Substance and Structure}
	
	My core project consists of three main parts, they are:
	\begin{itemize}
		\item The DHT
		\item The 3D World
		\item The Graphical Client
	\end{itemize}
	I have researched a number of potential implementations for the DHT including Kademlia~\cite{kademlia} and OpenDHT~\cite{odht} (which has an open source implementation with convenient Python bindings). I plan to implement my own based on the Kademlia whitepaper~\cite{kademlia}, potentially with some minor extensions (i.e. extra RPCs, for example: I may need a \texttt{GENERATE} RPC to generate a world chunk). This will be used as the basis for my distributed world.
	
	The 3D world will be a simple voxel-based one. With clients able to place and remove voxels from the world. The world will consist of a number of discrete `chunks' of voxels. These chunks will re represented as a simple 3D array and will represent a contiguous $x*y*z$ compartment of the world (where $z$ is the height of the world). This is the approach taken in games such as Minecraft~\cite{mc}. The world will also allow `entities' which are not part of the voxel array but are objects which can move freely in the 3D space (e.g. the player). Chunks will have a separate list to store any entities in that chunk along with their position. It might be worthwhile to explore usage of an kd-tree (or equivalent) for storage of entities but this would be a minor extension and is not required.
	
	Usage of the DHT to aid in locating entities as well as chunks in the network will be necessary, as when a client connects to the world looking for a particular player to take control of, they will need a way to find which chunk that player is currently in. In the event that the player is in no chunk (i.e. either the chunks they were in have been lost or they are new), then a random chunk can be selected as a spawn location.
	
	As this is a distributed system I will need some form of synchronisation between nodes; a logical ordering system such as a vector clock is unlikely to be appropriate as the system is operating in real time and the real time between events is relevant to the computation. So I intend to use NTP for synchronisation, as minor clock skew will not affect the system to a significant degree. I will be using an existing NTP library for this as it is likely to be more robust than any implementation I can produce.
	
	Additionally, there will be a client able to connect to this distributed 3D world and to interact with it. This will be a simple game client which displays the world and allows basic interaction with the world (likely little more than placing and destroying voxels). This could be implemented in OpenGL~\cite{opengl} either directly in C++ or using a library such as LWJGL~\cite{lwjgl}. Alternatively, it could be written using an existing engine such as Unity.
	
	Finally, I will need to write a very simple test agent which connects to the server and performs some example gameplay activity, such as constructing a basic structure. This will be needed to test the performance of the system in the evaluation phase. This will be a paired down version of the above client so will require little work.
	
	\section{Evaluation and Success Criteria}
	
	\subsection{Methodology}
	
	This project's success will be gauged entirely quantitatively, no human participants will be required.\\
	My project will be evaluated in the following ways:
	\begin{itemize}
		\item My DHT implementation will be compared to the Kademlia specification~\cite{kademlia}. The RPCs set out in the Kademlia paper will be analysed to ensure they function as specified, for example, the number of nodes contacted during a lookup should be $\mathcal{O}(\log{n})$ where $n$ is the number of nodes in the system.
		\item I will containerise the peer-to-peer server and deploy it on a small to medium scale. I will then test it by connecting a number of my test agents to the server to see how it performs. I will be testing both the network performance and the performance of the individual machines. These figures will be used as a baseline for the next section.
		\item I will then attempt to estimate how well the system will perform at scale by extrapolating from the baseline acquired in the previous section. I will use some higher level reasoning to compare the performance of my system with large contemporary systems such as the games World of Warcraft~\cite{wow}, EVE Online~\cite{eve} and Runescape~\cite{rune}.
		\item I will analyse my implementation's resilience to a number of attacks/failures. Potential examples include: Byzantine Faults, Sybil Attacks, Impersonation Attacks, etc.
	\end{itemize}

	\subsection{Success Criteria}
	
	\begin{enumerate}
		\item My DHT must adhere to the Kademlia specification. It is possible I will need to make some changes to fit the specification better to my needs and this is acceptable.
		\item The peer-to-peer node program must join the network, bootstrapping via some known node, and then will be able to participate in hosting the game world as it becomes part of the DHT.
		\item It must be possible to interact with the world using a simple 3D graphical client, which is able to place and remove voxels from the world. These changes must persist.
		\item The system must handle player moving between separate chunks (and thus, separate peers) seamlessly, with no loading screen.
		\item There must be a simple test agent which connects to and interacts with the world in some notional way to emulate the behaviour of a human user. This is for the purposes of my above quantitative evaluation methodology.
	\end{enumerate}
	
	\section{Possible Extensions}
	
	Below are a number of potential extensions to my project. This list is not exhaustive and I may choose to implement others as well as or instead of those listed below.
	
	\begin{itemize}
		\item \textbf{Weak Consensus/Voting System:} as mentioned in my introduction I propose the inclusion of a weak consensus or voting system to aid in the detection of rogue peers and, similarly, rogue clients. This would work by replicating each chunk over $n$ nodes where $n>=3$ and having each of the n peers review the activities of the others to ensure that they are acting correctly. If a large enough nodes were able to agree that a node is acting maliciously, they could blacklist it from the network.
		\item \textbf{3D Terrain Generation:} rather than having a bland, flat world, it would be nice to have some variable terrain. This could easily be achieved using Perlin noise~\cite{perlin}, a type of gradient noise commonly used in video games for realistic terrain generation.
		\item \textbf{Further World Features:} including other gameplay, for example a basic mob~\cite{mob}, would enable me to more accurately test the performance of the system at a large scale.
	\end{itemize}
	
	\section{Timetable}
	
	A brief outline of my plan of work is below.
	
	\begin{enumerate}
		\item \textbf{24th October to 13th November:} Study and implement the Kademlia DHT to specification~\cite{kademlia}.\\
		\textbf{Deliverable:} DHT which can be demonstrated with a toy system and tested with appropriate unit tests, must conform (with appropriate deviation) to the Kademlia specification.
		\item \textbf{14th November to 4th December:} Server-side 3D world implementation. This will involve setting up the state representation for the 3D world; making a server able to load a chunk from disk and then to complete the necessary processing to update the game state while a player is using to this chunk (processing will include, physics calculations such as collisions, handling interaction with the voxels and -- if implemented -- processing mobs). It will also need to be possible for a user to connect to the server and load a chunk which is stored in that node. The protocol for interaction between client and server will need to be established here.\\
		\textbf{Deliverable:} 3D world server, this will be the software run by a single peer in the network and will be built on top of the DHT.
		\item \textbf{5th December to 22nd December:} Implement 3D graphical client. This will be a simple 3D client used to interact with the world, written, most likely, in OpenGL. The client will connect to the DHT, locate the relevant chunk(s) to be rendered, load these by connecting to the relevant server(s) and provide an interactable display of the world. It will make use of the above protocol for communication between client and server.\\
		\textbf{Deliverable:} Graphical client program, able to join the P2P game by bootstrapping via a known node.
		\item \textbf{2nd January to 10th:} Implement test agent. This should be a fairly easy task as it will be a paired down version of the client developed earlier in the vacation. This is to be used in the evaluation stage and will have no graphical interface by default (though one may be pertinent for testing purposes).\\
		\textbf{Deliverable:} Test agent program, to be used to test the system at a small to medium scale.
		\item \textbf{11th January to 15th January:} Begin evaluation testing of system on a small scale using the test agent developed in the previous section.\\
		\textbf{Deliverable:} Baseline test data.
		\item \textbf{16th January to 29th January:} If time allows, then implement extensions which are feasible in remaining time. Additionally, write progress report demonstrating completion of project core and evaluation data acquired in previous section.\\
		\textbf{Deliverable:} Progress report and implementation of extension(s) if time allows.
		\item \textbf{30th January to 5th February:} Work on chapters introduction and preparation, adjust code from evaluation data and further work on extensions.\\
		\textbf{Deliverable:} Introduction and preparation chapter drafts and further work on extensions.
		\item \textbf{6th February to 12th February:} Work on implementation chapter and further work on extensions.\\
		\textbf{Deliverable:} Implementation chapter draft.
		\item \textbf{12th February to 19th February:} Work on evaluations and conclusions chapters.\\
		\textbf{Deliverable:} Entire dissertation first draft to be submitted to supervisor and DoS by March 20th.
		\item \textbf{Remaining Time:} Refining dissertation and, time permitting, further extension work.\\
		\textbf{Deliverable:} Dissertation final draft to be submitted to supervisor and DoS by April 20th.\\
		\textbf{Deliverable:} Dissertation due on 8th May.
	\end{enumerate}
	
	\section{Resource Declaration}
	
	I will be using my own computer for development, I am about to get a new computer, so I cannot comment on the exact specification yet, but I will be using the new one for the majority of the project. In the event of this failing, I have several fallback machines. Firstly my current laptop running Ubuntu 18.04 is still operational. Additionally, I have a desktop PC which is also with me at university which I can use in the event of both other computers failing. Furthermore, in the unlikely event that all of these fail I can use the MCS machines in the Computer Lab, my college computer room and my college library.
	
	I will be using Git for version control of my project and dissertation, these will be regularly pushed to GitHub. Additionally, I will be pulling the GitHub repository automatically to an external server I possess on a regular basis and also to MCS machines on a regular basis.
	
	I will also be making use of my own dedicated server for some of the testing of the project. This is an OVH dedicated server hosted in Roubaix, France. In the event of this failing, I propose to use several Raspberry Pis for my testing. If my server proves to be insufficient then I will be using AWS services supplied by Professor Crowcroft for testing.
	
	\begin{thebibliography}{9}
		\bibitem{wowstat} ``World of Warcraft subscriptions fall to 7.7 million''. \url{https://www.gameinformer.com/b/news/archive/2013/07/26/world-of-warcraft-subscriptions-fall-to-7-7-million.aspx}. Accessed: 2019-10-16.
		\bibitem{shard} ``Sharding'' on Wikipedia. \url{https://en.wikipedia.org/wiki/Shard_(database_architecture)}. Accessed: 2019-10-15.
		\bibitem{down} ``Club Penguin is shutting down''. \url{https://techcrunch.com/2017/01/31/club-penguin-is-shutting-down/}. Accessed: 2019-10-15.
		\bibitem{peerreview} Miller, J. Distributed virtual environment
		scalability and security, chapter 5. \url{https://www.cl.cam.ac.uk/techreports/UCAM-CL-TR-809.pdf}. Accessed: 2019-10-15.
		\bibitem{kademlia} Maymounkov, P. and Mazières, D. Kademlia: A Peer-to-peer Information System Based on the XOR Metric. \url{https://pdos.csail.mit.edu/~petar/papers/maymounkov-kademlia-lncs.pdf}. Accessed: 2019-10-16.
		\bibitem{odht} OpenDHT. \url{https://github.com/savoirfairelinux/opendht}. Accessed: 2019-10-16.
		\bibitem{mc} ``Chunk'' on the Minecraft Wiki. \url{https://minecraft.gamepedia.com/Chunk}. Accessed: 2019-10-17.
		\bibitem{opengl} OpenGL. \url{https://www.opengl.org/}. Accessed: 2019-10-17.
		\bibitem{lwjgl} LWJGL: Lightweight Java Game Library. \url{https://www.lwjgl.org/}. Accessed: 2019-10-17.
		\bibitem{perlin} ``Perlin Noise'' on Wikipedia. \url{https://en.wikipedia.org/wiki/Perlin_noise}. Accessed: 2019-10-17.
		\bibitem{mob} ``Mob'' on Wikipedia. \url{https://en.wikipedia.org/wiki/Mob_(gaming)}. Accessed: 2019-10-17.
		\bibitem{wow} World of Warcraft. \url{https://worldofwarcraft.com/}. Accessed: 2019-10-17.
		\bibitem{eve} EVE Online. \url{https://www.eveonline.com/}. Accessed: 2019-10-17.
		\bibitem{rune} Runescape. \url{https://www.runescape.com/}. Accessed: 2019-10-17.
	\end{thebibliography}
\end{document} 
	%TC:endignore
\end{document}